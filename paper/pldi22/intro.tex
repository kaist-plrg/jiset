\section{Introduction}\label{sec:intro}

In the beginning, JavaScript was a scripting language designed in a ten-day
hack. However, it has become a de facto Web standard and eventually becomes one
of the dominating programming languages in various fields. For example, Node.js
introduced full-stack JavaScript by supporting server-side programming, and
JavaScript has recently begun to be used even in embedded systems and
microcontrollers for the Internet of Things (IoT). Furthermore, according to the
annual report of GitHub\footnote{\url{https://octoverse.github.com/}},
JavaScript has consistently been the most popular programming language based on
the number of contributors to GitHub projects.

Recently, JavaScript has rapidly evolved with a yearly release cadence and open
development process. In 2015, The Ecma Technical Committee 39 (TC39) decided to
annually release ECMAScript, the standard specification for JavaScript, with a
massive update in ECMAScript 2015 (ES6, 2015)~\cite{es6}. Moreover, they
published the specification as an open-source project in a GitHub repository to
quickly adapt users' demands to the language.

In this fast-evolving nature of JavaScript, the need for language design
assistants (LDAs)~\cite{lda} has grown to understand original or revised
semantics correctly in the revision process. Historically, researchers have
proposed several language design assistants, such as LISA~\cite{lisa} and
$\asfsdf$~\cite{asf-sdf, meta-env}, with semantics editors and the generators
for language-based tools, such as parsers, compilers, interpreters, and so on.
However, only small-size domain-specific languages are their targets due to
their low expressive power. On the other hand, there are several attempts to
develop language design assistants for JavaScript in the academy and industry.
For example, in the academy, \citet{jsexplain} presented JSExplain, a JavaScript
reference interpreter written in OCaml that closely follows ECMAScript and
produces execution traces for debugging both ECMAScript and a given program. In
the industry, Mozilla and several contributors introduced JavaScript
interpreters written in pure JavaScript, such as Narcissus~\cite{narcissus} and
engine262~\cite{engine262}, to aid the development of new features in
JavaScript.

However, existing language design assistants for JavaScript have two
limitations: \textit{manual updates} and \textit{lack of connection with
conformance tests}. First, all existing assistants are developed manually; thus,
developers should manually update them for each frequent evolution of
ECMAScript. This manual approach is a tedious, labor-intensive, and even
error-prone process. Moreover, they lack information about the relationship
between specific semantics with conformance tests. When a language designer
desires to understand specific language semantics, one possible approach is to
learn them from related conformance tests. Fortunately, TC39 provides Test262,
an official conformance test suite for the latest version of ECMAScript.
However, existing assistants do not utilize them to give a better comprehension
of language semantics.

In this paper, we present $\tool$, a \textit{live} language design assistant for
JavaScript, which automatically synchronizes with a given language specification
and shows related conformance tests in the revision process. To synchronize our
tool with ECMAScript, we utilize another tool $\jiset$~\cite{jiset}, which
automatically extracts a mechanized specification from a given version of
ECMAScript. Note that a specification is mechanized if and only if it is both
executable and formalized.
\todo
