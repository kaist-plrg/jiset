\section{Evaluation}\label{sec:eval}

\begin{table*}
  \centering
  \caption{The analysis precision of $\tool$ without refinement
  (\stextsf{no-refine}), with refinement (\stextsf{refine}), and their
  difference ($\Delta$)}
  \label{table:precision}
  \resizebox{\textwidth}{!}{%
    \begin{tabular}{c|c?r@{~}c@{~}r@{~}l@{}r|r@{~}c@{~}r@{~}l@{}r?r@{~}c@{~}r@{~}l@{}r|r@{~}c@{~}r@{~}l@{}r?r@{~}c@{~}r@{~}l@{}r|r@{~}c@{~}r@{~}l@{}r}
      \multirow{2}{*}{\textbf{Checker}} &
      \multirow{2}{*}{\textbf{Bug Kind}} &
      \multicolumn{30}{c}{
        \textbf{Precision = (\# True Bugs) / (\# Detected Bugs)}
      }\\\cline{3-32} &&
      \multicolumn{10}{c?}{no-refine} &
      \multicolumn{10}{c?}{refine} &
      \multicolumn{10}{c}{$\Delta$}\\

      \specialrule{.1em}{.0em}{.0em}

      \myrowc
      {\mr{2}{\stextsf{Reference}}}
      {\me{2}{60}{125}{48.0}} {\me{2}{61}{98}{62.2}} {\me{2}{+1}{-27}{+14.2}}
      {\mr{1}{\stextsf{UnknownVar}}}
      {\me{1}{17}{81}{21.0}} {\me{1}{17}{53}{32.1}} {\me{1}{}{-28}{+11.1}}

      \myrowh
      {\mr{1}{\stextsf{DuplicatedVar}}}
      {\me{1}{43}{44}{97.7}} {\me{1}{44}{45}{97.8}} {\me{1}{+1}{+1}{+0.1}}

      \hline

      \smyrowc
      {\mr{1}{\stextsf{Arity}}}
      {\mr{1}{\stextsf{MissParam}}}
      {\me{1}{4}{4}{100.0}} {\me{1}{4}{4}{100.0}} {\me{1}{}{}{}}

      \hline

      \smyrowc
      {\mr{1}{\stextsf{Assertion}}}
      {\mr{1}{\stextsf{Assertion}}}
      {\me{1}{5}{57}{8.8}} {\me{1}{5}{31}{16.1}} {\me{1}{}{-26}{+7.4}}

      \hline

      \myrowc
      {\mr{2}{\stextsf{Operand}}}
      {\me{2}{22}{113}{19.5}} {\me{2}{22}{44}{50.0}} {\me{2}{}{-69}{+30.5}}
      {\mr{1}{\stextsf{NoNumber}}}
      {\me{1}{2}{65}{3.1}} {\me{1}{2}{6}{33.3}} {\me{1}{}{-59}{+30.3}}

      \myrowh
      {\mr{1}{\stextsf{Abrupt}}}
      {\me{1}{20}{48}{41.7}} {\me{1}{20}{38}{52.6}} {\me{1}{}{-10}{+11.0}}

      \specialrule{.1em}{.0em}{.0em}

      \mysrowh
      {\textbf{Total}}
      {91 / 299 (30.4\%)}{92 / 177 (52.0\%)}{+1 / -122 (+21.5\%)}

    \end{tabular}
  }
  \vspace*{-1.5em}
\end{table*}


We implemented $\tool$ as an open-source tool~\footnote{The link is anonymized
because of a double-blind review process} in Scala by extending $\jiset$, a
JavaScript IR-based semantics extraction toolchain, with a worklist-based
fixpoint algorithm for type analysis.  Since $\jiset$ cannot automatically
compiles algorithm steps written in an uncommon writing style to $\ires$
instructions, our tool only reports type-related specification bugs detected in
fully compiled abstract algorithms.  For built-in libraries, we only targeted
abstract algorithms of essential built-in objects: \jscode{Array},
\jscode{Object}, \jscode{Function}, \jscode{Math}, \jscode{Proxy}, and objects
for JavaScript primitive types.

To evaluate $\tool$, we answer the following research questions:
\begin{itemize}
  \item RQ1. \textbf{(Performance)} How long does $\tool$ take to perform type
    analysis for JavaScript specifications?
  \item RQ2. \textbf{(Precision)} How many type-related specification bugs
    detected by $\tool$ are true bugs?
  \item RQ3. \textbf{(Effect of Refinement)} Does the condition-based refinement
    improve the analysis precision with endurable performance degradation?
  \item RQ4: \textbf{(Detection of New Bugs)} Does $\tool$ detect new
    specification bugs in the latest version of ECMAScript?
\end{itemize}
The draft of the next version of ECMAScript (ES12, 2021) is fixed on March 9,
2021.  Thus, we targeted all different 864 versions existed in the official
ECMAScript repository~\footnote{https://github.com/tc39/ecma262} for the recent
three years from January 1, 2018 to March 9, 2021.  We performed our experiments
on five Ubuntu machines equipped with 4.2GHz Quad-Core Intel Core i7 and 32GB of
RAM.


\subsection{Performance}\label{sec:performance}

Figure~\ref{fig:stat} shows four statistics of type analysis using $\tool$ for
864 versions of ECMAScript: (a) the number of analyzed functions, (b) the number
of flow- and type-sensitive views, (c) the number of worklist iterations, and
(d) the analysis time.  For each version of ECMAScript, \inred{1,689.6}
functions (\stextsf{analyzed}) are analyzed and \inred{1,570.0} functions among
them are fully compiled (\stextsf{compiled}) on average.  Since ECMAScript has
gradually evolved for three years, \inred{1,357} out of \inred{1,490} analyzed
functions are fully compiled at the first version in 2018 but \inred{1741} out
of \inred{1,863} analyzed functions are fully compiled in the latest version.
$\tool$ divides functions to flow- and type-sensitive views.  On average,
\inred{91.2}K views existed for each version and \inred{53.9} views existed for
each function.

We measured performance of $\tool$ based on the number of iterations for
worklist algorithm and the analysis time.  For each version of ECMAScript,
$\tool$ took \inred{136.0} seconds with \inred{298.8}K iterations of worklist
algorithm on average.  The average analysis time consists of \inred{8.0} seconds
for the specification extraction (\stextsf{extract} with gray bars),
\inred{127.0} seconds for the type analysis (\stextsf{analyze} with blue bars),
\inred{0.7} seconds for the bug detection (\stextsf{detect} with red bars).


\subsection{Precision}\label{sec:precision}

% \begin{table}
%   \centering
%   % \includegraphics[width=\columnwidth]{img/author}
%   \caption{The creators and resolvers of true bugs.}
%   \label{fig:author}
%   \resizebox{\columnwidth}{!}{$
%     \inred{
%     \begin{array}{c?r|r|r|r|r|r?c?c}
%       \textbf{Rank} >
%       & 0 & 10 & 20 & 30 & 40 & 50
%       & \multicolumn{1}{c?}{\multirow{2}{*}{\textbf{Total}}}
%       & \multicolumn{1}{c}{\multirow{2}{*}{\textbf{Unknown}}}\\\cline{1-7}
%       \textbf{Rank} \leq
%       & 10 & 20 & 30 & 40 & 50 & 60
%       & & \\
% 
%       \specialrule{.1em}{.0em}{.0em}
% 
%       \textbf{\# Created}
%       & 23 & 4 & 1 & 14 & 2 & 1
%       & 45 & 46\\\hline
%       \textbf{\# Resolved}
%       & 57 & 6 & 12 & 2 & 0 & 0
%       & 77 & 14\\
% 
%     \end{array}
%     }
%   $}
%   \vspace*{-1em}
% \end{table}

\begin{figure}
  \centering
  \begin{subfigure}[b]{0.24\textwidth}
    \includegraphics[width=\textwidth]{img/ttl-chro}
    \caption{TTLs sorted by creation time.}
  \end{subfigure}
  \begin{subfigure}[b]{0.24\textwidth}
    \includegraphics[width=\textwidth]{img/ttl-count}
    \caption{The histogram of TTLs.}
  \end{subfigure}
  \caption{Time to Lives (TTLs) of true bugs.}
  \vspace*{-1.5em}
  \label{fig:ttl}
\end{figure}

We measured the analysis precision based on the number of true bugs in
type-related specification bugs detected by $\tool$.  As described in the
\stextsf{refine} column of Table~\ref{table:precision}, the analysis precision
is \inred{52.0}\% because it detected \inred{177} type-related bugs and
\inred{92} bugs are true bugs among them.  The reference checker detected the
largest number of bugs with \inred{62.2}\% precision; it found \inred{61} true
reference bugs for \inred{17} unknown variables (\stextsf{UnknownVar}) and
\inred{44} already defined variables (\stextsf{DuplicatedVar}). The arity
checker and the assertion checker found \inred{4} missing parameters
(\stextsf{MissParam}) with \inred{100.0}\% precision and \inred{5} assertion
failures (\stextsf{Assertion}) with \inred{16.1}\% precision, respectively.
Finally, the operand checker detected two different kinds of wrong typed operand
bugs: \inred{2} non-numeric operand bugs (\stextsf{NoNumber}) for numeric
operators with \inred{33.3}\% precision and \inred{20} unchecked abrupt
completion bugs (\stextsf{Abrupt}) with \inred{52.6}\% precision.

To understand the severity of detected true bugs, we extended $\tool$ to
automatically extract when they are created and resolved in the ECMAScript
official repository.  We defined that a bug is \textit{created} when it exists
in a certain version but does not exist in the right before version, it is
\textit{resolved} vice versa, and the \textit{Time to Live (TTL)} of a bug is
how many days they last.  Figure~\ref{fig:ttl} describes TTLs of true bugs; the
left chart (Figure~\ref{fig:ttl}(a)) depicts their periods sorted by their
creation time, and the right chart (Figure~\ref{fig:ttl}(b)) depicts the
histogram of TTLs in a logarithmic scale.  Among \inred{92} true bugs,
\inred{47} bugs are \textit{inherited}, which means that they are created before
2018, and \inred{14} bugs still existed in the latest ECMAScript thus they are
newly detected bugs by $\tool$.  We will discuss details of \inred{14} newly
found bugs in Section~\ref{sec:new-bug}.  Even though we assume that \inred{47}
inherited bugs are created in January 1, 2018, the average TTL is \inred{409.5}
and the maximum TTL is \inred{1,164}.  All bugs having the maximum TTL are
inherited and newly detected bugs.

We investigated \inred{85} false-positive bugs to reason about why they are
detected by $\tool$.  Among them, \inred{30} bugs are caused by the failed
extraction of mechanized specification because of wrong writing styles.  For
example, ECMAScript is written in HTML and $\jiset$ extracts abstract algorithms
based on \code{emu-alg} HTML tags.  However, several abstract algorithms are
defined with the opening tag \code{<emu-alg>} but without the closing tag
\code{</emu-alg>}.  Thus, $\jiset$ fails to extract them and it causes
false-positive bugs.  The remaining \inred{55} bugs are caused by imprecise type
analysis results.  Among them, \inred{20} bugs are caused by imprecise analysis
for conditions of assertions and branches strongly related with specific
function calls.  For example, consider the following algorithm step in the
algorithm \textbf{GetValue}:
\begin{figure}[H]
  \centering
  \vspace*{-0.5em}
  \includegraphics[width=0.7\columnwidth]{img/adv-refine-example}
  \vspace*{-0.5em}
\end{figure} \noindent
Since the algorithm \textbf{IsUnresolvableReference} always returns
$\code{false}$ when the $\code{Base}$ field of a given reference record is
$\nconst{unresolvable}$, the field access \code{V.Base} never be
$\nconst{unresolvable}$ on line 4.a.  However, the abstract environment in type
analysis only stores that the variable $\code{V}$ has the $\code{Reference}$
type thus $\nconst{unresolvable}$ is also passed as the argument of
\textbf{ToObject}.  We believe that more advanced refinement technique can
resolve this problem by pruning out infeasible field types depending on specific
contexts.



\subsection{Effectiveness of Refinement}\label{sec:effect-refine}

\begin{table*}
  \centering
  \caption{Type-related specification bugs newly detected by $\tool$ in the
  official draft of ECMAScript 2021 (ES12).}
  \label{table:new-bug}
  \resizebox{\textwidth}{!}{%
  \begin{tabular}{@{}c@{~}?c|@{~}c@{~}|l|c|@{~}c@{~}|@{~}r@{}}
    \multicolumn{1}{@{}c?}{\textbf{Name}} &
    \multicolumn{1}{c}{\textbf{Feature}} &
    \multicolumn{1}{@{}c@{~}}{\textbf{\#}} &
    \multicolumn{1}{c}{\textbf{Description}} &
    \multicolumn{1}{@{~}c@{~}}{\textbf{Checker}} &
    \multicolumn{1}{@{}c}{\textbf{Created}} &
    \multicolumn{1}{@{}c@{~}}{\textbf{TTL}}\\\specialrule{.1em}{.0em}{.0em}

    ES12-1 &
    Switch &
    3 &
    \makecell[l]{
        Variables \code{hasDuplicates} and \code{hasUndefinedLabels} are already \\
        defined in algorithms for \jscode{case} blocks of \jscode{switch}
        statements.
    } &
    \stextsf{Reference} &
    2015-09-22 &
    1,996 days\\\hline

    ES12-2 &
    Try &
    3 &
    \makecell[l]{
        Variables \code{hasDuplicates} and \code{hasUndefinedLabels} are already \\
        defined in algorithms for \jscode{try} statements.
    } &
    \stextsf{Reference} &
    2015-09-22 &
    1,996 days\\\hline

    ES12-3 &
    Arguments &
    1 &
    \makecell[l]{
        A variable \code{index} is already defined in
        \textbf{CreateMappedArgumentsObject}.
    } &
    \stextsf{Reference} &
    2015-09-22 &
    1,996 days\\\hline

    ES12-4 &
    Array &
    2 &
    \makecell[l]{
        A variable \code{succeeded} is already defined in algorithms for
        \jscode{Array} objects.
    } &
    \stextsf{Reference} &
    2015-09-22 &
    1,996 days\\\hline

    ES12-5 &
    Async &
    1 &
    \makecell[l]{
        A variable \code{value} is already defined in \textbf{Evaluation} for
        \jscode{yield} expressions.
    } &
    \stextsf{Reference} &
    2015-09-22 &
    1,996 days\\\hline

    ES12-6 &
    Class &
    1 &
    \makecell[l]{
        A variable \code{ClassHeritage} is not defined in \textbf{Contains}
        \\ for tails of \jscode{class} declarations.
    } &
    \stextsf{Reference} &
    2015-09-22 &
    1,996 days\\\hline

    ES12-7 &
    Branch &
    1 &
    \makecell[l]{
        A variable \code{Statement} is not defined in \textbf{EarlyErrors} for
        \jscode{if} statement.
    } &
    \stextsf{Reference} &
    2015-09-22 &
    1,996 days\\\hline

    ES12-8 &
    Arguments &
    2 &
    \makecell[l]{
        Abrupt completions are used in \textbf{DefineOwnProperty} and
        \textbf{GetOwnProperty} \\ for \jscode{arguments} objects without any
        checks.
    } &
    \stextsf{Operand} &
    \inred{2015-09-22} &
    \inred{1,996 days}\\
  \end{tabular}
  }
  \vspace*{-1.5em}
\end{table*}

\begin{figure}
  \centering
  \begin{subfigure}[b]{0.24\textwidth}
    \includegraphics[width=\textwidth]{img/compare-iter}
    \caption{The histogram of iterations.}
  \end{subfigure}
  \begin{subfigure}[b]{0.24\textwidth}
    \includegraphics[width=\textwidth]{img/ratio-iter}
    \caption{The ratio of iterations.}
  \end{subfigure}
  \begin{subfigure}[b]{0.24\textwidth}
    \includegraphics[width=\textwidth]{img/compare-time}
    \caption{The histogram of time.}
  \end{subfigure}
  \begin{subfigure}[b]{0.24\textwidth}
    \includegraphics[width=\textwidth]{img/ratio-time}
    \caption{The ratio of time.}
  \end{subfigure}
  \caption{The comparison of iterations and analysis time without refinement
  (\stextsf{no-refine}) and with refinement (\stextsf{refine}).}
  \label{fig:performance-compare}
  \vspace*{-1.5em}
\end{figure}

We evaluated the effectiveness of condition-based refinement based on how much
it improved analysis precision without significant performance degradation.  We
performed $\tool$ without refinement (\stextsf{no-refine}) for 864 versions of
ECMAScript, and compared their analysis results with the analysis with
refinement (\stextsf{refine}).

For performance, Figure~\ref{fig:performance-compare} describes the comparison
of iterations and analysis time without refinement and with refinement; the left
charts (Figure~\ref{fig:performance-compare}(a) and
\ref{fig:performance-compare}(c)) are histograms of iterations and analysis
time, and the right charts (Figure~\ref{fig:performance-compare}(b) and
\ref{fig:performance-compare}(d)) are scatter charts for their ratio.  Without
refinement, type analysis took \inred{90.9} seconds with \inred{259.1}K
iterations on average.  For each version, the number of iterations is increased
at least \inred{0.99}x, at most \inred{1.36}x, and \inred{1.16}x on average, and
the analysis time is increased at least \inred{1.05}x, at most \inred{1.41}x,
and \inred{1.98}x on average.

Table~\ref{table:precision} shows the analysis precision without refinement,
with refinement, and their difference.  The refinement improved the analysis
precision from \inred{30.4}\% to \inred{52.0}\% by removing \inred{122}
false-positive bugs and to detect \inred{one} more true bug.  Among six
difference bug kinds, the most significant improvement is about non-numeric
operand bugs (\stextsf{NoNumber}) from \inred{3.1}\% to \inred{33.3}\%
by removing \inred{59} false-positive bugs.  The refinement technique
successfully prunes out non-numeric values in conditions for numeric types.
The refinement also significantly increased the analysis precision for unknown
variable bugs (\stextsf{UnknownVar}) and assertion failures
(\stextsf{Assertion}) by removing \inred{28} and \inred{26} false-positive bugs,
respectively.  However, there is no effect of refinement to detect missing
parameter bugs (\stextsf{MissParam}) because it is possible to precisely analyze
possible callee functions for each function invocation without condition-based
refinement.


\subsection{Detection of New Bugs}\label{sec:new-bug}

Among \inred{92} true bugs detected by $\tool$, \inred{14} bugs are newly
detected and still exist in the latest version of ECMAScript.
Table~\ref{table:new-bug} summarizes the bugs categorized by their kinds and
related JavaScript language features and it also contains their TTLs. We
manually investigated their exact creation time if they are created before
2018.  Except two bugs in ES12-8, all bugs were introduced in the initial
commit of the open development on September 22, 2015.  Thus, 12 newly detected
bugs last for 1,966 days until March 9, 2021.  The other two bugs in ES12-8 were
created when a contributor introduced the prefixes \textbf{?} and \textbf{!}
on December 16, 2015 thus it last for 1,910 days.  We reported the newly
detected bugs to TC39; all of them are confirmed by the committee and will be
fixed in ECMAScript 2022 (ES13).

ES12-1 contains three bugs that are due to the duplicated definition of
variables in three different syntax-directed algorithms for \jscode{case} blocks
of \jscode{switch} statements: \code{hasDuplicates} in
\textbf{ContainsDuplicateLabels}, and \code{hasUndefinedLabels} in
\textbf{ContainsUndefinedBreakTarget} and
\textbf{ContainsUndefinedContinueTarget}.  A \jscode{case} block optionally
contains \jscode{case} clauses.  In the beginning of three algorithms,
\code{hasDuplicates} (or \code{hasUndefinedLabels}) is defined if the clauses
exist.  However, the exactly same variable is defined again after this
conditional steps.  It means that three algorithms for \jscode{case} blocks
having \jscode{case} clauses always have already defined variable bugs for
\code{hasDuplicates} (or \code{hasUndefinedLabels}).  Similarly, ES12-2 also
contains three bugs caused by the exactly same reason in same abstract
algorithms for \jscode{try} statements.

The bug in ES12-3 is a reference bug for an already defined variable
\code{index} in the abstract algorithm \textbf{CreateMappedArgumentsObject}.
For each function call in JavaScript programs, an \jscode{arguments} object is
created by \textbf{CreateMappedArgumentsObject}.  In the algorithm, the variable
\code{index} is defined to handle the index of a given list of arguments.
However, the variable is defined twice in step 14 and 17 of the algorithm.

ES12-4 contains two reference bugs for already defined variable \code{succeeded}
in \textbf{DefineOwnProperty} of \jscode{Array} objects and
\textbf{ArraySetLength}.  The \jscode{Array} objects are exotic, which means
that they are not ordinary objects and have special algorithms for specific
behaviors.  The two algorithms are wrapper algorithms of
\textbf{OrdinaryDefineOwnProperty} which updates object properties.  They define
the variable \code{succeeded} to represent the result of
\textbf{OrdinaryDefineOwnProperty}.  However, the variable is defined twice in a
specific condition.

The bug in ES12-5 is a reference bug for already defined variable \code{value}
in \textbf{Evaluation} of defined \jscode{yield} expressions.  In the evaluation
of \jscode{yield * e}, the variable \code{value} is defined twice to
represent 1) the evaluation result of the given expression \jscode{e} in step
3, and 2) to the iterator value in step 7.c.viii.1.

The bug in ES12-6 is a reference bug for unknown variable \code{ClassHeritage}
in \textbf{Contains} for tails of \jscode{class} declarations.  A tail of
\jscode{class} declaration consists of an optional class heritage with
\jscode{extends} keyword and a class body enclosed by curly braces.  When the
optional class heritage does not exist, the variable \code{ClassHeritage} is not
defined but the \textbf{Contains} algorithm tries to access it without any check
of its existence.

The bug in ES12-7 is a reference bug for unknown variable \code{Statement} in
\textbf{EarlyErrors} for \jscode{if} statements.  In syntax-directed algorithms,
the ordinal numbers are used as prefixes of variables when multiple sub-ASTs are
produced by same productions.  A \jscode{if} statement contains two sub-ASTs
produced by the \textit{Statement} production thus the ordinal number prefixes
are required for the variable \code{Statement}.  However, the
\textbf{EarlyErrors} algorithm for \jscode{if} statements uses it without any
ordinal number prefixes.

ES12-8 contains two operand type bugs related to abrupt completions in
\textbf{DefineOwnProperty} and \textbf{GetOwnProperty} for \jscode{arguments}
objects.  The two algorithms define or get own properties of \jscode{arguments}
exotic objects.  They utilize the \textbf{Get} algorithm that returns JavaScript
values stored in object properties or abrupt completions.  Thus, they should
check whether results of \textbf{Get} are abstract completions or not before
using them.  However, they use the results without any abrupt completion checks.
