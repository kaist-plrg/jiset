\begin{abstract}
JavaScript was initially designed for client-side programming in web browsers,
but its engine is now embedded in various kinds of host software.  Despite the
popularity, since the JavaScript semantics is complex especially due to its
dynamic nature, understanding and reasoning about JavaScript programs are
challenging tasks.  Thus, researchers have proposed several attempts to define
the formal semantics of JavaScript based on ECMAScript, the official JavaScript
specification.  However, the existing approaches are manual, labor-intensive,
and error-prone and all formal semantics target former versions of ECMAScript
5.1 (ES5.1).  It is critical in understanding modern JavaScript language
features introduced after ECMAScript 6 (ES6). Moreover, ECMAScript has been
annually released since 2015, which made already five updates after ES5.1.

To alleviate the problem, we propose \( \tool \), a JavaScript IR-based
Semantics Extraction Toolchain. It is the first tool that \textit{automatically
synthesizes} parsers and AST-IR translators directly from a given ECMAScript.
For syntax, we develop a parser generation technique with \textit{lookahead
parsing} for \( \bnfes \), a variant of the extended BNF used in ECMAScript.
For semantics, \( \tool \) synthesizes AST-IR translators using \textit{forward
compatible} rule-based compilation. Each \textit{compile rule} describes how to
convert each step of abstract algorithms written in structured natural language
into \( \ires \), an Intermediate Representation that we designed for
ECMAScript. For the five most recent ECMAScript versions, \( \tool \) succeeds
to synthesize parsers for all versions, and to compile \inred{XX.XX}\% of
algorithm steps in average. We manually complete the core part of formal
semantics of the next version of ECMAScript (ES11) based on the extracted
semantics and it passed all \inred{XX,XXX} applicable tests. Based on the formal
semantics, we found \inred{X} specification errors in ES11 and all of them were
confirmed by TC 39.  Furthermore, we also successfully applied \( \tool \) to
\inred{X} feature proposals in Stage 3.
\end{abstract}
