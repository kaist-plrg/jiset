\documentclass[sigconf,review,anonymous=true]{acmart}

\usepackage{kotex}
\usepackage[T1]{fontenc}
\usepackage{beramono}
\usepackage{listings}
\usepackage{xcolor}
\usepackage{qtree}
\usepackage{colortbl}
\usepackage{ulem}
\usepackage{graphicx}
\usepackage{caption}
\usepackage{subcaption}
\usepackage{listings}
\usepackage{enumitem}
\usepackage{algorithm}
\usepackage[noend]{algpseudocode}
\usepackage{booktabs}

% color
\definecolor{gainsboro}{rgb}{0.86, 0.86, 0.86}
\newcommand*{\belowrulesepcolor}[1]{%
  \noalign{%
    \kern-\belowrulesep
    \begingroup
      \color{#1}%
      \hrule height\belowrulesep
    \endgroup
  }%
}
\newcommand*{\aboverulesepcolor}[1]{%
  \noalign{%
    \begingroup
      \color{#1}%
      \hrule height\aboverulesep
    \endgroup
    \kern-\aboverulesep
  }%
}

% dashed line
\usepackage{array}
\usepackage{arydshln}
\setlength\dashlinedash{0.2pt}
\setlength\dashlinegap{1.5pt}
\setlength\arrayrulewidth{0.3pt}

% removed ACM references
\settopmatter{printacmref=false}

% Scala code style
\definecolor{dkgreen}{rgb}{0,0.6,0}
\definecolor{gray}{rgb}{0.5,0.5,0.5}
\definecolor{mauve}{rgb}{0.58,0,0.82}
\lstdefinestyle{myScalastyle}{
  frame=tb,
  language=scala,
  aboveskip=3mm,
  belowskip=3mm,
  showstringspaces=false,
  columns=fixed,
  basicstyle={\footnotesize\ttfamily},
  numbers=none,
  keywordstyle=\color{blue},
  commentstyle=\color{dkgreen},
  stringstyle=\color{mauve},
  frame=single,
  breaklines=true,
  breakatwhitespace=true,
  tabsize=3,
}

% ECMAScript Intermediate Reprentations
\lstdefinestyle{ires}{
  frame=tb,
  aboveskip=3mm,
  belowskip=3mm,
  showstringspaces=false,
  columns=fixed,
  basicstyle={\footnotesize\ttfamily},
  numbers=none,
  keywordstyle=\color{blue},
  commentstyle=\color{dkgreen},
  stringstyle=\color{mauve},
  frame=single,
  breaklines=true,
  breakatwhitespace=true,
  tabsize=3,
}

% JavaScript code style
\lstdefinelanguage{JavaScript}{
  keywords={typeof, new, true, false, catch, function, return, null, catch, switch, var, if, in, while, do, else, case, break},
  keywordstyle=\color{blue}\bfseries,
  ndkeywords={class, export, boolean, throw, implements, import, this},
  ndkeywordstyle=\color{darkgray}\bfseries,
  identifierstyle=\color{black},
  sensitive=false,
  comment=[l]{//},
  morecomment=[s]{/*}{*/},
  commentstyle=\color{purple}\ttfamily,
  stringstyle=\color{red}\ttfamily,
  morestring=[b]',
  morestring=[b]"
}

\lstdefinestyle{myJSstyle}{
  language=JavaScript,
  extendedchars=true,
  basicstyle=\footnotesize\ttfamily,
  showstringspaces=false,
  showspaces=false,
  numbers=none,
  tabsize=2,
  breaklines=true,
  showtabs=false,
  captionpos=b
}

% load macros
% small textsf
\newcommand{\stextsf}[1]{\textsf{\small #1}}

% code styles
\definecolor{dkgreen}{rgb}{0,0.6,0}
\lstdefinelanguage{JavaScript}{
  keywords={let, async, await, break, case, catch, class, const, continue,
    debugger, default, delete, do, else, enum, export, extends, false, finally,
    for, function, if, import, in, instanceof, new, null, return, super, switch,
    this, throw, true, try, typeof, var, void, while, with, yield},
  keywordstyle=\color{blue}\bfseries,
  ndkeywordstyle=\color{darkgray}\bfseries,
  identifierstyle=\color{black},
  sensitive=false,
  comment=[l]{//},
  morecomment=[s]{/*}{*/},
  commentstyle=\color{dkgreen}\ttfamily,
  stringstyle=\color{purple}\ttfamily,
  morestring=[b]',
  morestring=[b]"
}
\lstdefinestyle{FigureJS}{
  language=JavaScript,
  numbers=left,
  stepnumber=1,
  extendedchars=true,
  basicstyle=\footnotesize\ttfamily,
  showstringspaces=false,
  showspaces=false,
  xleftmargin=\parindent,
  tabsize=2,
  breaklines=true,
  showtabs=false,
  captionpos=b
}

% tool name
\newcommand{\bigtool}{\textsf{JSTAR}}
\newcommand{\tool}{\textsf{JSTAR}}

% jiset
\newcommand{\jiset}{\textsf{JISET}}

% ires
\newcommand{\ires}{{\text{IR}_\text{ES}}}

% comments with red color
\newcommand{\inred}[1]{{\color{red}{#1}}}

% codes
\newcommand{\jscode}[1]{\text{\lstinline[style=FigureJS,numbers=none]!#1!}}
\newcommand{\code}[1]{\texttt{#1}}

% abstraction
\newcommand{\abs}[1]{{#1}^\sharp}

% initial
\newcommand{\init}[1]{{#1}_\iota}

% states
\newcommand{\elem}{d}
\newcommand{\aelem}{\abs{\elem}}
\newcommand{\iaelem}{\elem_\iota^\sharp}


\begin{document}

\title{\( \tool \): JavaScript IR-based Semantics Extraction Toolchain}

\author{Jihyeok Park}
\email{jhpark0223@kaist.ac.kr}
\affiliation{\institution{KAIST}}

\author{Jihee Park}
\email{j31d0@kaist.ac.kr}
\affiliation{\institution{KAIST}}

\author{Sukyoung Ryu}
\email{sryu.cs@kaist.ac.kr}
\affiliation{\institution{KAIST}}

\begin{abstract}
JavaScript was initially designed for client-side code in web browsers,
but its engine is now embedded in various kinds of host software.
Despite its popularity, since the JavaScript semantics is complex
especially due to its dynamic nature, understanding and reasoning
about JavaScript programs are challenging tasks.  Thus,
researchers have proposed several attempts to define the formal semantics
of JavaScript based on ECMAScript, the official JavaScript specification.
However, the existing approaches are manual, labor-intensive, and
error-prone, and they all target only the ECMAScript 5.1 version.
This problem is critical in understanding recent JavaScript programs
because ECMAScript has been annually released since 2015, which made
already five updates after ECMAScript 5.1.

To alleviate the problem, we propose \( \tool \), a JavaScript IR-based Semantics
Extraction Toolchain.  It \textit{automatically} extracts the JavaScript syntax and
semantics from ECMAScript specifications.  \( \tool \) generates a JavaScript
parser for a given grammar written in \( \bnfes \), an extended BNF
used in ECMAScript specifications.  It also converts each
abstract algorithm written in a structured natural language into
functions written in \( \ires \), an Intermediate Representation
for ECMAScript specifications.  The compilation from algorithms to
\( \ires \) functions is defined with \textit{compile rules}.
We applied \( \tool \) to the next proposed version, ECMAScript 2020,
and the automatically extracted semantics passed all \inred{XXXXX} tests 
in test262, the official ECMAScript conformance test suite.
\inred{Moreover, \( \tool \) also successfully extracts
semantics for \inred{XX} proposed features by adding only \inred{XX}
minor compile rules.}
\end{abstract}

\keywords{
  JavaScript,
  mechanized formal semantics,
  recursive descent parsing
}

\maketitle

\section{Introduction}

JavaScript is one of widely used programming languages not only for client-sides
but also for server-side programming~\cite{???} and even operating systems~\cite{???}.
However, developers suffer from the intricate semantics of JavaScript and
such complexities cause security vulnerabilities. For example, the following JavaScript
code seems to be tautology:
\begin{lstlisting}[style=myJSstyle]
function f(x) { return x == !x; }
\end{lstlisting}
Unfortunately, it returns \( \code{false} \) when the argument is an empty array
\( \code{[]} \). To correctly understand such situations, the understanding of
formal semantics of JavaScript is necessary.

\begin{figure}
  \centering
  \includegraphics[width=0.5\textwidth]{img/previous.png}
  \caption{Previous approaches of semantics extractions for ECMAScript}
  \label{fig:previous}
\end{figure}

However, the existing JavaScript foraml semantics are fully manually written
by researchers based on ECMAScript specifications. ECMAScript is a standardized language
of JavaScript and its specifications formally define its syntax and semantics.
As described in Figure~\ref{fig:previous}, former researchers implement parser and
describe semantics using their own intermediate representation (IR) according to the ECMAScript
specifications. However, their approaches are fully manual thus excessively labor-intensive
to handle full semantics.

Moreover, it is not suitable approach to cope with the updates of ECMAScript specifications.
Most of full formal semantics of ECMAScript only supports ECMAScript 5.1 that was
released in December 2009. Until 2014 years, it was not big deal because ECMAScript
specifications are rarely updated. However, in late 2014, The Ecma Technical
Committee 39 (TC39) decided to release ECMAScript on a yearly cadence. From the
ECMAScript 6, officially ECMAScript 2015, they updated specifications every years
in June. Now, no researches correctly deal with recent ECMAScript features;
lexical bindings (\( \code{let} \)), spread operator (\( \code{...} \),
classes, for-of operators, async functions, or generators.

To alleviate such problems, we propose \textit{Automatic Semantics Extractor (ASE)}
for ECMAScript specifications. The tool automatically generates JavaScript parser
from syntax provided by ECMAScript specifications. It also extracts formal
JavaScript semantics in semi-automatic ways. It is semi-automatic because
it depends on manually defined \textit{conversion rules}. Each rule describes how abstract
algorithms in specifications are converted into functions of our core language \( \ires \).
However, to save the labor of defininig such rules, we also propose \textit{rule generation assistant}
that suggests new conversion rules based on not yet convertable algorithms.
The assistant help us deal with changed or newly introduced algorithms defined in new version of
ECMAScript specifications as well.

Our main contribution is our tool ASE that mechanizes extraction of JavaScript syntax and semantics
from ECMAScript specifications:
\begin{itemize}
  \item ASE enables to extract the syntax and semantics of ECMAScript 2019,
    which is the latest released version of ECMAScript in June 2019.
    Moreover, we successfully applied our tool into the next proposed version,
    ECMAScript 2020. The official conformance test suite, test262, provides
    \inred{XXXXX} tests for ECMAScript 2020 and we passed \inred{XXXXX} tests
    among them.
  \item ASE is also applicable in-process proposed langauge features in a modular way.
    ECMAScript specifications are open-source based documentations. Thus, anyone
    could propose some new language feautres. Then, the committee carefully inspects
    such proposals and adds them into specifications. Each in-process proposed language
    feature has its own specification and tests. Thus, we apply ASE into the specification
    and evalute the corresponding tests. Among \inred{XX} proposals, we successfully
    pass all the tests in \inred{XX} proposals and partially pass the tests in \inred{XX}
    proposals.
  \item Moreover, all extraction mechanisms in ASE is automatic thus we could use this tool
    to find possible specification errors. We found \inred{X} possible specification errors
    in ECMAScript 2019 and reported them into TC39, which is the Ecma Technical Committee.
    We confirm that all of them are real specification errors.
\end{itemize}

\section{Overview}

% TODO
% \begin{figure}
%   \centering
%   \includegraphics[height=6em]{img/???}
%   \caption{}
%   \label{fig:array-literal}
% \end{figure}

\begin{itemize}
  \item Sytnax: automatically generate parser from ES-BNF
    \begin{itemize}
      \item ES-BNF features
      \item Packrat Parsers with 1-lookahead
      \item Example
    \end{itemize}
  \item Semantics: automatically convert abstract algorithms into Core programs
    \begin{itemize}
      \item Abstract algorithms
      \item Example
    \end{itemize}
\end{itemize}
\begin{itemize}
  \item Figure of overall structure
  \item Explanation of each module
\end{itemize}

\section{Parser Generation}

In this section, we introduce a way to automatically generate JavaScript parsers
from given ECMAScript specifications. First, we explain \( \bnfes \), an extension
of BNF used in ECMAScript to describe lexical and syntactic grammars of JavaScript.
We defined a recursive descent parser generator to support \( \bnfes \) notations.
We implemented our idea by extending the basic Scala parser combinator.
The implementation also support the automatic semicolon insertion,
one of the most distinct parsing feautres in ECMAScript.

\subsection{\( \bnfes \): Grammars for ECMAScripts}

ECMAScript provides their lexical and syntactic grammars as grammars
based on \( \bnfes \), an extension of Backus-Naur form for ECMAScript.
\( \bnfes \) consists of a number of \textit{productions}
with the following forms:
\[
  \nt(\param_1, \cdots, \param_k) ::=
  (\cond_1 \Rightarrow)^? \rhs_1 \mid
  \cdots \mid
  (\cond_n \Rightarrow)^? \rhs_n
\]
The left-hand side represents a parametric nonterminal \( \nt \) with
multiple boolean parameters \( \param_1, \cdots, \param_n \).
A production has multiple right-hand sides with optional conditions.
The condition \( \cond \) is either \( \param_i \) or \( ! \param_i \),
the negation of \( \param_i \).
For example, consider the following production:
\[
  \nt(\param) ::= \param \Rightarrow \code{a}
  \mid \; !\param \Rightarrow \code{b}
  \mid  \code{c}
\]
Then, \( \nt(\kwt) \) means \( \code{a} \mid \code{c} \)
and \( \nt(\kwf) \) means \( \code{b} \mid \code{c} \).

Each right-hand side \( \rhs \) is a sequence of the following symbols:
\begin{itemize}
  \item \( \boxed{\symb} \):
    basic symbols; terminal \( \code{x} \) or
    non-terminal \( \nt(\argument_1, \cdots, \argument_k) \)
  \item \( \boxed{\symb?} \): optional symbols
  \item \( \boxed{\pm \symb} \): positive/negative lookahead symbols
  \item \( \boxed{\symb \butnot \symb'} \): exclusive symbols
  \item \( \boxed{\nolt} \): no line terminator symbols
\end{itemize}

Basic symbols are either terminal or non-terminal.
A non-terminal symbol \( \nt(\argument_1, \cdots, \argument_k) \)
has multiple parameters and each argument \( \argument_i \) is
a boolean value \( \kwt \), \( \kwf \) or a parameter \( \param_i \).
An optional symbol \( \symb? \) is same with \( \epsilon \mid \symb \)
where \( \epsilon \) denotes the empty sequence.
A positive(negative) lookahead symbol \( +\symb \)(\( -\symb \))
checks that the symbol \( \symb \) succeeds(fails) and
\textit{never consumes any input}.
The exclusive symbols \( \symb \butnot \symb' \)
first checks that the symbol \( \symb \) succeeds
and then checks that the result does not correspond to \( \symb' \).
The no line terminal symbol \( \nolt \) is a special symbol
that restricts the white spaces between two other symbols.

\subsection{First/Follow Packrat Parsing}

Our goal is to automatically generate JavaScript parsers from ECMAScript
grammars written in \( \bnfes \). However, there are several problems
to use traditional lex/yacc style parser generator.
\begin{itemize}
  \item \textbf{Context-Sensitive Tokens} ECMAScript tokens are context-sensitive
    because of the regular expressions and template strings.
    For example, the code \( \code{/x/g} \) might be a regular expression token
    or four tokens to represent divided by the variable \( \code{x} \) and \( \code{g} \).
\end{itemize}

% There are two different parsing approaches:
% ``top-down'' approach (LL-style parsing) and
% ``bottom-up'' approach (LR-style parsing).
% Top-down approach support 
% 
% There are two different kinds of parsers: top-down parsing, bottom-up parsing.
% 
% 
% 
% Such features could be easily represented in Parsing Expression Grammars (PEGs).
% 
% 
% 
% Backus–Naur form(BNF)에 다섯 가지의 방식을 추가한 형태로 lexical 및 syntactic grammar를 제공한다.
% 각각의 
combination of recursive descent parsers with predictive parsers

first terms and follow terms

\subsection{Generating Parsers for \( \bnfes \)}
how to handle each \( \bnfes \) features
% \begin{table}
%   \centering
%   \[
%     \begin{array}{l|l}
%       \bnfes & \peg
%       \\\hline
%       \code{T}_\code{opt} & \code{T?}
%       \\\hline
%       \code{T but not T'} & \code{-T'} \; \code{\textasciitilde} \; \code{T}
%       \\\hline
%       \code{[lookahead} = \code{T]} & \code{+T}
%       \\\hline
%       \code{[lookahead} \neq \code{T]} & \code{-T}
%       \\\hline
%       % \code{T[P}^\code{*}\code{]} & \text{parameters}
%       % \\\hline
%       % \begin{itemize}
%       %   \item \code{?P} & \text{passing argument} \code{P}
%       %   \item \code{+P} & \kwtrue
%       %   \item \code{\textasciitilde}\code{P} &  \kwfalse
%       % \end{itemize}
%       % \code{[+P] T} & \text{if} \; \code{P} \; \text{is} \; \kwtrue \;\text{, added} \; \code{T}
%       % \\\hline
%       % \code{[\textasciitilde}\code{P] T} & if  \code{P}  is  \kwfalse , added  \code{T}
%       % \\\hline
%     \end{array}
%   \]
%   \caption{Conversion from \( \bnfes \) into \( \peg \)}
%   \label{table:bnfes}
% \end{table}

% \begin{figure}
%   \centering
%   \[
%     \begin{array}{rcrcl}
%       Call[Await] &::=& 
%       A(p, q, r) &::=& && a A(p, \kwt, \kwf)\\
%                  &\mid& p &Rightarrow& b\\
%     \end{array}
%   \]
%   \caption{Example of \( \bnfes \)}
%   \label{figure:bnfes}
% \end{figure}

% code size 줄이기 위해서 모두 풀어헤치지 않고 함수로 변환했다.

% PEG를 그대로 사용하면 backtrack을 이미 성공한 곳에서는 하지 않기 때문에
% 생기는 문제를 CFG로 적혀있는 것을 해결하기 위해서 lookahead를 추가함
% Lexical Grammar는 context-sensitive하기 때문에 먼저 lexer를 돌리지 못함
% gll에서 비슷하게 first/follow를 한 것을 보고 PEG parser를 extend함

\section{Algorithm Compiler}

\begin{figure}
  \centering
  \includegraphics[width=0.5\textwidth]{img/algo_compiler.png}
  \caption{Overall structure of the algorithm compiler.}
  \label{fig:algo-compiler}
\end{figure}

In this section, we propose the \textit{algorithm compiler}
that compiles abstract algorithms into \( \ires \) functions
and its overall structure is described in Figure~\ref{fig:algo-compiler}.
Before compiling abstract algorithms,
the tokenizer first tokenizes each abstract algorithm into
a list of tagged tokens. Then, the token parser constructs
syntactic parse tree of the given token list.
Finally, the \( \ires \) function generator constructs the coressponding
\( \ires \) function from the given parse tree.
The parser and the generator is dependent on conversion rules
and we define the general conversion rules for abstract algorithms in
ECMAScript specifications. Like Coq, a proof assistant, we also provide
the \textit{rule generation assistant} to make easy write conversion rules.
It diagnoses root causes of failed parsing tokens
and suggests new conversion rules based on
statistical analysis of algorithm steps that failed to be parsed.

\subsection{Tokenizer}

Abstract algorithms in ECMAScript specifications are written in structured
natural languages in HTML files. An algorithm consists of ordered steps
and it might contain sub-steps as well. For example,
the \( \code{ToPrimitive} \) algorithm in Figure~\ref{fig:to-primitive}
has three steps and second step has seven sub-steps.
Moreover, the tokens of each step has its own HTML tag and each tag
has the following meaning:
\[
  \begin{array}{c|l}
    \text{HTML tags} & \text{meanings}\\\hline\hline
    \code{<var>} & \text{variables}\\\hline
    \code{<emu-grammar>} & \text{productions}\\\hline
    \code{<emu-nt>} & \text{non-terminal syntax}\\\hline
    \code{<code>} & \text{ECMAScript codes}\\\hline
    \code{<emu-const>} & \text{constant values}\\\hline
    \code{<emu-val>} & \text{values}\\\hline
    \code{<ol>} & \text{ordered sub-steps}\\\hline
    \code{<ul>} & \text{unordered sub-steps}\\\hline
    \code{<sup>} & \text{superscripts}\\\hline
    \text{otheriwse} & \text{simple texts}\\\hline
  \end{array}
\]
We try to keep the tag information for each token to generate more
precise conversion rules. For example, if and only if a token has a tag
\( \code{<var>} \), it is a parameter or a local variable.
Thus, it is possible to construct a conversion rule precisly discriminates
identifiers and other components.

The tokenizer first recognizes the overall structures of steps.
Then, it divides each step into sequence of tagged tokens.
If an HTML element has a explicit tag, it is converted into a single token
with the tag. Otherwise, it is splitted into multiple tokens and each token
should be a sequence of alphanumeric characters or a single
non-alphanumeric character. For example, in the \( \code{ToPrimitive} \)algorithm,
\( \textbf{\code{"default"}} \) is a single token with the tag \( \code{<code>} \)
and \( \code{@@toPrimitive} \) is splitted into three text tokens
\( \code{@} \), \( \code{@} \), and \( \code{toPrimitive} \).

For the linear structures, the tokenizer flatten the structured steps into
a single list. Some conversion rules should handle multiple steps such as
the conditional statements (if-then-else). Thus, we decide to break down
structured algorithms using three special tokens;
\( \tend \) denotes the end of a single step,
\( \tin \) and \( \tout \) the start and the end of nested steps, respectively.
For example, the \( \code{ToPrimitive} \) algorithm is tokenized as follows.
\[
  \begin{array}{l}
    \code{Assert} \cdots \tend\\
    \code{If} \cdots \tin \code{If} \cdots \tend
    \cdots \code{Return} \cdots \tend \tout \tend\\
    \code{Return} \cdots \tend\\
  \end{array}
\]

\subsection{Token Parser}

The token parser parses a given list of tokens into a syntactic parse tree.
It depends on the given \textit{conversion rules} that consists of
two parts parsing rules and mapping from parsing rules into corresponding
\( \ires \) components.
A parsing rule consists of basic token matchers and other parsing rules
including itself. For example, the following parsing rule is simplifed version for
sub-step \( i \) in the \( \code{ToPrimitive} \) algorithm.
\begin{lstlisting}[style=myScalastyle]
// statements
val Stmt = "Let" ~ varT ~ "be" ~ Expr ~ "."
// expressions
val Expr = (
  // identifiers
  varT |
  // return if abrupt
  "?" ~ Expr |
  // function calls
  textT ~ "(" ~ repsep(varT, ",") ~ ")" |
  // lists
  "<<" ~ repsep(Expr, ",") ~ ">>"
)
\end{lstlisting}

Each parsing rule is written in extended Scala parser combinators.
We modify the meaning of altenative composition operator ( \( | \) ) to collect
all longest matched results. If the parser detects that a step cannot be
parsed with the given parsing rules. It reports that a step is not possible
to be parsed under the given rules into the rule generation assistant.
Moreover, even though the parser could parse a step, it will be also reported
if it could be parsed in not a single but multiple ways.
Each string literal is a token matcher for tokens without any tags and checks
that the token has same string value with the sring literal.
The \( \code{varT} \) and \( \code{textT} \) is also token matchers
for token with the tag \( \code{<var>} \) and no tags.
The \( \code{Step} \) and \( \code{Expr} \) are user-defined parsing rules
defined with our extended parser combinators.
Morevoer, it supports all helpers functions defined in Scala parser combinators.
For example, the helper function \( \code{repsep(p, q)} \) generates a new
parsing rule that denotes zero or more repetitions of the parsing rule \( \code{p} \)
using another parsing rule \( \code{q} \) as separators.
Finally, the token parser with the above rules parses sub-step \( i \) in the
\( \code{ToPrimitive} \) into the following syntactic parse tree:
\includegraphics[width=0.5\textwidth]{img/parse_tree.png}

\subsection{\( \ires \) Function Generator}

The \( \ires \) function generator takes a parse tree from the token parser
and generates an \( \ires \) function. The given conversion rules describe
the mapping from each parsing rule into corresponding \( \ires \) components.
The \( \ires \) is an imperative style programming languages that we proposed
in order to represent each step of abstract algorithms in ECMAScript specifications.

% IR_ES 더 설명하기

\subsection{Rule Generation Assistant}

\section{Building entire system to evaluate code}\label{sec:framework}

In this section, we will demonstrate how our idea can be implemented to evaluate programming language. Although it is hard to generalize the process because the format of specification or 
environment of the programming language is varying, We shows the possibility to build entire system for evaluating target language with minimal human effort by giving end-to-end example of ECMAScript language.

\subsection{Extracting specification}
First step is to collect language specification and extract component that we can process automatically. For our target language, ECMAScript, has HTML format of specification in their website, and
in the specification they have BNF form of syntax, highly structured algorithm steps, and verbose texts to describe additional environment to evaluating ECMAScript source code.
We get these information by parsing HTML document, and passing these result to our previous module to get AST, Parser, and executable algorithms.

To organize these primitives and running the program, we have to model initial environment of language such as initialization of builtin types, or entry point of the program, etc.

\subsection{Generating initial environment for evaluating}
For ECMAScript Specification, initial environment consists of the Builtin objects, Intrinsic objects. We will describe how to generate each component.

\subsubsection{Builtin objects}

 Builtin object is pre-defined object with many built-in functions. For example, In ECMAScript \( \code{Array} \) refers an object which can generate an array by
 calling it. Each Builtin Object provides fields and function, just as ordinary object. When its function is called,
 the corresponding algorithm specified in the document should be called. 
 
 In specification, a built-in function object usually matches to abstract algorithm. Its algorithm name
 describes the hierarchy about object it should be contained. For example, there is an abstract algorithm named \( \code{ Object.prototype.propertyIsEnumerable } \). With its name,
 we can know that it corresponds to builtin function in the object \( \code{ Object.prototype } \) as attribute. 
 To support builtin object, we make structure of builtin object and its attributes.
 After we correctly organizes structure, then we made algorithm that automatically map corresponding field to the abstract algorithm in specification by matching its function name.
 
 \subsubsection{Intrinsic objects}

 While builtin objects has canonical name to refer itself in ECMAScript language, It is also frequently referred in specification. But name of builtin objects
 are defined for reference in Host language, so it needs another name to refer it in specification language. For example, \( \code{Array} \) is a name for
 ECMAScript reference, so abstract algorithm should call \\
 \( \code{realm.[[GlobalObject]].[[GetOwnProperty]]("Array")} \) to get ECMAScript \( \code{Array} \) object. Instead of this verbose way, abstract algorithm introduces Intrinsic objects, which
are alias for built-in objects. For example it is referred as \( \code{\%Array\%} \) to refer Array Object.
 Map between Intrinsic name and Builtin objects are available in specification as table form in section "6.1.7.4 Well-Known Intrinsic Objects", so we parse this table to generate global identifier to object.

\subsection{Algorithm scope}

 In specification, All algorithm has name to refer it, but some algorithms have same names and they are called in context-dependent way.
In "5.2 Algorithm Conventions", it defines three types of abstract operation.
 \begin{itemize}
  \item Abstract operation: they can be called by name in any context
  \item Method-like abstract operation: algorithm dispatch depends on receiver object
  \item Syntax-directed operation: algorithm dispatch depends on receiver AST
  \end{itemize}

First, we categorized sections by types of algorithm.
Method-like abstract operation has separate sections to describe it, so we manually add these sections to treat differently.
Also Syntax-directed operations are described in distinct sections, so both case is easy to extract it.
Rest algorithms are considered as ordinary Abstract operation.

\subsubsection{Abstract operation}

We just convert abstract operations to function, and bind global identifier to that function to call it by algorithm name.

\subsubsection{Method-like abstract operation}

In ECMAScript specification, there are some objects which has similar interface, but they have different algorithms. 

For Example, Ordinary Object and String Exotic Object is different object in specification, but they have same name of algorithm(i.e. \( \code{[[getOwnProperty]](N)} \))

In this case, algorithm dispatching should consider receiver Object(i.e. \( \code{x.[[GetOwnProperty]](N)} \)).
For method-like abstract operation, we organized map from algorithm name to function for each type,
 and we insert functions as property of object when such type of object is created. So In our example, when  \( \code{x} \) is created with Ordinary Object type,
 then we insert attributes with name \( \code{[[GetOwnProperty]]} \) and function from method-like abstract operation "[[GetOwnProperty]]" in Ordinary Object Internal method section, at the time of map creation.

 \subsubsection{Syntax-directed operation}
 
 Like Method-like abstract operation, many AST Value (in specification, it is called Parse Node) uses same name of algorithm but the content is different.
The most frequent example is Evaluation of the AST Value. Evaluation is described as \( \code{result of evaluating Expression} \), but Non-terminal
\( \code{ Expression } \) did not has specific algorithm \( \code{ Evaluation } \). Instead of this, some non-terminal that can be derived from \( \code{ Expression } \) has \( \code{ Evaluation }\),
(i.e. \( \code{ CallExpression } \)). Also in Syntax-directed operation, the sub-component of AST can be used as values. For example, In the
Evaluation algorithm of\\ \( \code{CallExpression : CallExpression Arguments} \), algorithm can uses \( \code{CallExpression}\) or \( \code{Arguments}\) to refer sub-component of AST.
 For Syntax-directed operation, we insert converted functions in AST structure, and add special expression to call Syntax-directed operation which requires AST Value and algorithm name.
We gather which AST can be applied to a function. Also, to use child node of AST, we passes child node of AST to function as argument.

% \begin{itemize}
%   \item \textbf{Global Algorithms} Global algorithms are possible to be accessed
%     by any other algorithms. The top-level algorithm \( \code{RunJobs} \) is also global.
%   \item \textbf{Type-dependent Algorithms} Several algorithms have different
%     bodies dependent on types of their owner data. In ECMAScript specifications,
%     there are two different data types; language types and specification types.
%     The language types are explicitly used in JavaScript languages such as
%     \( \code{Boolean}, \code{Number}, \code{String} \), and \( \code{Object} \).
%     However, specification types are only used in specifications such as
%     Lists, Records, and Lexical Environments. Among them, some data types
%     have their own member algorithms. For example, language type \( \code{Object} \)
%     has several internal methods described in abstract algorithms. It depends
%     on 
% 
%     several data types have their own member algorithms. For example, 
%     
%     For example, the internal method
%     \( \code{[[Call]]} \) has different semantics dependent on its owner object.
%     Moreover, the owner objects are used in the body of algorithms as well.
%     Thus, we extend the parameters to accept its owner object as an implicit argument.
%   \item \textbf{Syntax-directed Algorithms} The semantics of each JavaScript syntax
%     are written in 
% \end{itemize}

% \subsection{semantic 컴파일하기 위해 하는 것 + specification 구조}
% 기본적인 원칙 :
% \begin{itemize} 
% \item 자바스크립트로 알고리즘 이름이랑 파라미터, 스텝을 가져와서 코어의 글로벌 ID에 매치 시켜줌
% \item initial Heap을 생성해줌
% \end{itemize}
% 매치를 시킨다는 뜻은 코어의 글로벌 ID에 코어 함수 형태로 알고리즘을 넣는 것임
% 

\section{Evaluation}\label{sec:eval}

We evaluate \( \tool \) based on the following four research questions:
\begin{itemize}
  \item RQ2) \textbf{Effectiveness} Is \( \tool \) effective to extract syntax and
    semantics of existing ECMAScript specifications?
  \item RQ1) \textbf{Correctness} Does \( \tool \) correctly extract JavaScript syntax
    and semantics from ECMAScript 2020 speicification?
  \item RQ3) \textbf{Adaptability} Does \( \tool \) deal with future proposed features?
  \item RQ4) \textbf{Specification Error Detection} What specification errors
    are detected by \( \tool \)?
\end{itemize}

\subsection{Effectiveness}

\begin{figure}[t]
  \centering
  \begin{subfigure}{0.23\textwidth}
    \includegraphics[width=\textwidth]{img/all-version-syntax.png}
    \caption{The parser generator.}
    \label{fig:syntax-all-version}
  \end{subfigure}
  \hfill
  \begin{subfigure}{0.23\textwidth}
    \includegraphics[width=\textwidth]{img/all-version-sem.png}
    \caption{The algorithm compiler.}
    \label{fig:semantics-all-version}
  \end{subfigure}
  \caption{The result of the parser generator and the algorithm compiler for
  ECMAScript specifications from 2016 to 2020.}
  \label{fig:all-version}
\end{figure}

We disucss the effectiveness of \( \tool \) by checking that the parser generator
and the algorithm compiler are able to handle existing ECMAScript specifications.
Figure~\ref{fig:all-version} shows the result of \( \tool \) from ECMAScript 2016
to 2020. While ECMAScript 5.1 is the first version that officially supports
the specification in HTML. However, two oldest versions ECMAScript 5.1 and 2015 are written
in quite different styles with recent versions including HTML tags used in abstract algorithms.
Thus, we evaluate only five recent versions from ECMAScript 2016 to 2020.
We first apply \( \tool \) into ECMAScript 2016 and counts how many productions in syntax
and algorithm steps in semantics are automatically extracted. For ECMAScript 2017 to 2020,
we consider only newly added productions and algorithm steps in ecah version.
For syntax, the parser generator successfully generates parsers for all \inred{XXX} productions
in ECMAScript 2016 and \inred{XXX} new productions in ECMAScript 2017 to 2020.
For semantics, the algorithm compilear successfully compiles \inred{XXXX} of \inred{XXXX} steps
into \( \ires \) instructions for ECMAScript 2016, and \inred{XXXX} of \inred{XXXX} newly added
steps are compiled in ECMAScript 2017 to 2020.

\subsection{Correctness}

\begin{table}[] \centering
  \begin{tabular}{lr}\toprule
    \belowrulesepcolor{gainsboro}
    \rowcolor{gainsboro} \textbf{All tests262 Tests} & \textbf{36,794} \\
    \aboverulesepcolor{gainsboro}\midrule
    Annexes/Internationalisation & 1,774\\ \hdashline
    In-process features & \inred{XXXX} \\\midrule
    Module tests & \inred{XXXX} \\\hdashline
    Non-strict tests & \inred{XXXX} \\\hdashline
    \belowrulesepcolor{gainsboro}
    \rowcolor{gainsboro} \textbf{ECMAScript 2020 Strict Script Tests} & \textbf{\inred{XXXXX}} \\
    \aboverulesepcolor{gainsboro}\midrule
    Not supported features & \inred{XXXX} \\\midrule
    \belowrulesepcolor{gainsboro}
    \rowcolor{gainsboro} \textbf{Applicable Tests} & \textbf{\inred{XXXXX}} \\
    \aboverulesepcolor{gainsboro}\midrule
    Passed tests & \inred{XXXX} \\\hdashline
    Failed tests & \inred{XXXX} \\\bottomrule
  \end{tabular}
  \caption{The syntax testing results of ECMAScript 2020 with test262.}
  \label{table:test262}
\end{table}

To check that \( \tool \) correctly extracts semantics, we apply our tool into ECMAScript 2020
specification. It is the next version of ECMAScript that is actively updated in GitHub~\ref{es2020}
to be released in the next year. Ecma Technical Committee 39 (TC39) officially provides
the conformance test suite, test262, for ECMAScript 2020. Thus, we develop \( \ires \)
interpreter to evaluate the \( \ires \) compiled from tests in test262.
However, test262 also provides tests for in-process features and other extensions
such as web browsers and internationalisation.
Thus, we exclude such tests from test262 test suite for evaluations.
Figure~\ref{table:test262} describes how to break down tests and the test results for parsers
and interpreters. We fixed the version of ECMAScript 2020 and test262 in August in 2019.
In that version, the test262 consists of 36,794 tests and we first exclude \inred{XXXX} tests
for extensions and in-process features. Then, we focus on only strict script not non-strict
or module codes thus we remove \inred{XXXX} tests. Finally, we have \inred{XXXX}


Then, we filter out tests having not supported features explained in Section~\ref{sec:framework}.
Table~\ref{table:}
We first exclude tests about not-yet applied proposed features, 

\subsection{Adaptability}
\subsection{Specification Error Detection}


% test262 결과를 어떻게? 전체 개수 / applicable tests 개수 / succes, fail 개수
% \inred{refactoring for compile rules / get statistics of compile rules}
% 개수만 - stmt / expr / cond / ref 몇개씩? / corner cases 개수
% \inred{proposed language features / related tests in test262 / diff of compile rules}
% 각 feature name / 의미 / compile 성공률 / 결과 test
% \inred{supprot "Assert: ... "?/ confirm found spec errors by TC39}

\section{Related Work}\label{sec:related}

\begin{itemize}
  \item JISET: JavaScript IR-based Semantics Extraction Toolchain~\cite{jiset}
\end{itemize}

\section{Conclusion}\label{sec:conclude}
The annual updates of the ECMAScript specification makes it difficult to
build program analysis or formal verification of JavaScript due to the
required human efforts in modeling a moving target.  In this paper,
we proposed a tool \( \tool \), which \textit{automatically} extracts the
syntax and semantics specified in the ECMAScript specification.  The tool
generates a parser from a language grammar written in \( \bnfes \), and
compiles abstract algorithms written in English to \( \ires \) functions.
The only parts that require manual efforts are compile rules and
language-specific global setting, which we provide for ECMAScript
as an example case study.  We also support a rule assistant to guide
how to convert new abstract algorithms.  We evaluated \( \tool \) by
applying it to ECMAScript 2020 with Test262.  The extracted semantics
passed all \inred{XXXXX} applicable tests in Test262 and we showed
that the tool is also applicable to an incomplete language proposal as
a case study.  The tool also detected six confirmed errors from the
specification and the proposal. The implementation of the tool is
publicly available as an open-source project.


\bibliographystyle{ACM-Reference-Format}
\bibliography{ref}

\end{document}
